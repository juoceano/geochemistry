\RequirePackage{fix-cm}  % Fix Font shape `OT1/cmr/m/n' size substitution.
%-----------------------------------------------------------------------------%
%	Packages & Other Configurations
%-----------------------------------------------------------------------------%
\documentclass[a4paper,10pt]{article}
\usepackage[top=1in, bottom=0.6in, left=1in, right=0.9in]{geometry}

\usepackage[utf8]{inputenc} %add acents
\usepackage{setspace} % command \doublespacing etc...
\usepackage{lineno} % number lines
\usepackage{epsf,epsfig} % includegraphics [pdf, png etc]
\usepackage{amsmath} %adicionei esse pacote pra vc poder usar o draft%
\usepackage{textcomp} %símbolos de texto
\usepackage{natbib} % bibtex - adicionar referencia
% \usepackage{url} % for bibtex - configuracoes de urls
\usepackage{tabularx} % for tables
\usepackage[hidelinks]{hyperref}  % Add URL links.
% \usepackage[bookmarks=false,colorlinks=true,urlcolor={green},linkcolor={green},pdfstartview={XYZ null null 1.22}]{hyperref} %all references


%-----------------------------------------------------------------------------%
%	Adicionar a Watermark
%-----------------------------------------------------------------------------%
\usepackage{draftwatermark}
\SetWatermarkAngle{45}
\SetWatermarkLightness{0.9}
\SetWatermarkFontSize{5cm}
\SetWatermarkScale{0.5}
\SetWatermarkText{Relatório Final}



%-----------------------------------------------------------------------------%
%	Informações sobre o PDF
%-----------------------------------------------------------------------------%

\pdfinfo{%
  /Title    (GEO117 - Relatório Final)
  /Author   (Ju Leonel)
  /Creator  (Ju Leonel)
  /Producer (Ju Leonel)
  /Subject  (Geoquímica - Aulas Praticas)
  /Keywords (geoquímica, projeto)}

%-----------------------------------------------------------------------------%
%	Documento
%-----------------------------------------------------------------------------%
\title{GEO117 - Práticas de Geoquímica - IGEO-UFBA}
\author{\vspace{-10ex}}
\date{\vspace{-10ex}}

\begin{document}

  \maketitle
  %\doublespacing
  \onehalfspace

  \begin{tabular*} {0.9\textwidth}{@{\extracolsep{\fill} } l l}
    \hline
    Professora: Juliana Leonel & Atendimento: Sextas-feiras \\
    E-mail: \href{mailto:jleonel@ufba.br}{jleonel@ufba.br} & Horário atendimento: 13:00-14:00 \\
    Aulas: Segundas-feiras & Local atendimento: IGEO - Sala 10 - 2\textsuperscript{o} andar\\
    Horário- Aulas: 13:00 - 15:45 & Homepage: \url{http://juoceano.github.io/geochemistry}\\
    \hline
  \end{tabular*}

  \vspace{3ex}

  \centerline{ \textbf{Relatório Final}}


  \section* {1. Formatação}
    \noindent

    a) Não há necessidade de capa (se houver capa, esta será contabilizada no limite de 5 páginas);

    b) Espaçamento 1.5 ao longo do texto com exceção de onde é explicitamente estipulado diferente no modelo que se encontra no final do documento;

    c) Fonte: Times New Romam

    d) Demais formatações estão descritas no final do documento.


  \section* {2. Conteúdo }
    \noindent

    a) Título do trabalho;

    b) Identificação dos membros do grupo;

    c) Resumo

    d) Introdução;

    e) Justificativa;

    f) Objetivos;

    g) Descrição da área de estudo;

    h) Resultados e discussão;

    i) Conclusões;

    j) Lista de referências bibliográficas usadas ao longo do texto.

   \section* {3. Avaliação}
   \noindent

    Abaixo estão listados alguns dos critérios de avaliação do trabalho.

    \textbf{1. Estrutura:} o trabalhho seguiu a formatação requisitada? Há lógica na apresentação e desenvolvimento das idéias?

    \textbf{2. Clareza:} as sentenças são concisas? O vocabulário usado é preciso? O leitor lê o texto facilmente e entende as ideias apresentadas pelos autores sem esforço?

    \textbf{3. Grafia:} há erros de gráfia, gramática ou pontuação?

    \textbf{4. Referencias bibligráficas:} as referências bibliográficas são corretamente citadas ao longo do texto? Todas as referências citadas no texto estão presentes na lista de bibliografia?

    \textbf{5. Relevância das informações:} todas as informações presentes tem relevância para o trabalho ou algumas (ou muitas) foram usadas apenas para aumentar o tamanho do texto?

    \textbf{6. Entendimento sobre o assunto:} o texto mostra conhecimento profundo e robusto por partes dos autores sobre o assunto?

    \textbf{7. Apresentação dos resultados:} Os dados são apresentados de forma clara? Os resultados são apresnetados de forma apropriada através do uso de tabelas e/ou gráficos?

    \textbf{8. Discussão:} As hipotéses usadas para explicar os resultados são bem estruturadas e embasadas em outros estudos?
    
    \textbf{9.Conclusão:} Expressa o que realmente concluiu-se do trabalho ou é apenas um resumo desse ou a retomada de tópicos da discussão?

\vspace{10ex}

*************************************************************************************

  {\centering
  \textbf{TÍTULO DO TRABALHO: EM MAIÚSCULA - MÁXIMO DE TRÊS LINHAS. TIMES NEW ROMAN 13, NEGRITO E  CENTRALIZADO}

  \textit{linha em branco}

  \textbf{Autores (Times New Romam 10, negrito e centralizado. Separar os autores por ponto e vírgula - Ex: Maria, J. M.; João, J. B.)}

  \textit{linha em branco}

  \par
  }

  \noindent
  \textbf{RESUMO} - Times New Romam 12, alinhado à , negrito e maiúsculo.

  (Times 12 justificada, máximo de 200 palavras)

  \vspace{1ex}
  \noindent
  \textbf{INTRODUÇÃO} - Times 12, alinhado à esquerda, negrito e maiúsculo.

  (Times new Roman 12, justificada)

  \vspace{1ex}
  \noindent
  \textbf{JUSTIFICATIVA} - Times 12, alinhado à esquerda, negrito e maiúsculo.

  (Times new Roman 12, justificada)

  \noindent
  \textbf{OBJETIVOS} - Times 12, alinhado à esquerda, negrito e maiúsculo.

  (Times new Roman 12, justificada)

  \vspace{1ex}
  \noindent
  \textbf{DESCRIÇÃO DA ÁREA DE ESTUDO} - Times 12, alinhado à esquerda, negrito e maiúsculo.

  (Times new Roman 12, justificada)

  \vspace{1ex}
  \noindent
  \textbf{RESULTADOS ESPERADOS} - Times 12, alinhado à esquerda, negrito e maiúsculo.

  (Times new Roman 12, justificada)

  \vspace{1ex}
  \noindent
  \textbf{REFERÊNCIAS BIBLIOGRÁFICAS} - Times 12, alinhado à , negrito e maiúsculo.

  \vspace{1ex}
  \noindent
  Waine, B.; Kent, C. (1969). Why Batman and Superman can never go to the beach. Proceedings from

  the Third International Conference of the Solitary Heroes Society, 5:304-309.

  (Times 11, justificada, espaçamento simples entre linhas)


%\clearpage %termina o texto e tudo que estiver flutuante se encaixa ali
%\newpage % começa uma nova página
%\bibliographystyle{chicago} % estilo que vai sair a bibliografia (posso usar outros)
%\bibliography{references_ju} %entre chaves vai o nome do arquivo de referencias

\end{document}
